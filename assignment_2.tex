%%%%%%%%%%%%%%%%%%%%%%%%%%%%%%%%%%%%%%%%%
% Beamer Presentation
% LaTeX Template
% Version 1.0 (10/11/12)
%
% This template has been downloaded from:
% http://www.LaTeXTemplates.com
%
% License:
% CC BY-NC-SA 3.0 (http://creativecommons.org/licenses/by-nc-sa/3.0/)
%
%%%%%%%%%%%%%%%%%%%%%%%%%%%%%%%%%%%%%%%%%

%----------------------------------------------------------------------------------------
%	PACKAGES AND THEMES
%----------------------------------------------------------------------------------------

\documentclass{beamer}

\mode<presentation> {

% The Beamer class comes with a number of default slide themes
% which change the colors and layouts of slides. Below this is a list
% of all the themes, uncomment each in turn to see what they look like.

%\usetheme{default}
%\usetheme{AnnArbor}
%\usetheme{Antibes}
%\usetheme{Bergen}
%\usetheme{Berkeley}
%\usetheme{Berlin}
%\usetheme{Boadilla}
%\usetheme{CambridgeUS}
%\usetheme{Copenhagen}
%\usetheme{Darmstadt}
%\usetheme{Dresden}
%\usetheme{Frankfurt}
%\usetheme{Goettingen}
%\usetheme{Hannover}
%\usetheme{Ilmenau}
%\usetheme{JuanLesPins}
%\usetheme{Luebeck}
\usetheme{Madrid}
%\usetheme{Malmoe}
%\usetheme{Marburg}
%\usetheme{Montpellier}
%\usetheme{PaloAlto}
%\usetheme{Pittsburgh}
%\usetheme{Rochester}
%\usetheme{Singapore}
%\usetheme{Szeged}
%\usetheme{Warsaw}

% As well as themes, the Beamer class has a number of color themes
% for any slide theme. Uncomment each of these in turn to see how it
% changes the colors of your current slide theme.

%\usecolortheme{albatross}
%\usecolortheme{beaver}
%\usecolortheme{beetle}
%\usecolortheme{crane}
%\usecolortheme{dolphin}
%\usecolortheme{dove}
%\usecolortheme{fly}
%\usecolortheme{lily}
%\usecolortheme{orchid}
%\usecolortheme{rose}
%\usecolortheme{seagull}
%\usecolortheme{seahorse}
%\usecolortheme{whale}
%\usecolortheme{wolverine}

%\setbeamertemplate{footline} % To remove the footer line in all slides uncomment this line
%\setbeamertemplate{footline}[page number] % To replace the footer line in all slides with a simple slide count uncomment this line

%\setbeamertemplate{navigation symbols}{} % To remove the navigation symbols from the bottom of all slides uncomment this line
}

\usepackage{natbib}
\usepackage{graphicx} % Allows including images
\usepackage{booktabs} % Allows the use of \toprule, \midrule and \bottomrule in tables

%----------------------------------------------------------------------------------------
%	TITLE PAGE
%----------------------------------------------------------------------------------------

\title[DMP]{Dynamic Movement Primitives} % The short title appears at the bottom of every slide, the full title is only on the title page

\author{Abhishek Padalkar} % Your name
\institute[HBRS] % Your institution as it will appear on the bottom of every slide, may be shorthand to save space
{
Hochschule Bonn-Rhein-Sieg \\ % Your institution for the title page
\medskip
\textit{abhishek.padalkar@smail.inf.h-brs.com} % Your email address
}
\date{\today} % Date, can be changed to a custom date

\begin{document}



\begin{frame}
\titlepage % Print the title page as the first slide
\nocite{*}
\end{frame}

\begin{frame}{Problem addressed}
\begin{itemize}

\item Exploring knowledge-base representation frameworks for DMPs which allow :
\begin{itemize}
\item Representation of DMP's weights, initial condition, goal and scaling parameters. 
\item Representation of information related to functionality of DMPs. 
\item Correct combination of DMPs to generate desired trajectory.
\end{itemize} 
\item Combining simple dynamic motion primitives for accomplishing a complex task using knowledge-base representation framework. 
\end{itemize}
\end{frame}

\begin{frame}{Relevance of the problem addressed}
\begin{itemize}
\item Dynamic motion primitives can be used in mobile robots for robust navigation. 
\item But learning entire motion as a single primitive reduces re-usability of that motion.  
\item A mobile robot can be benefited from this work as it can use DMP framework for navigation by combining simple motion primitives to navigate along a complex path.
\end{itemize}
\end{frame}

\begin{frame}{Advantages of Dynamic Motion Primitves}
\begin{itemize}

\item DMP learns trajectories in terms of attractor landscape of non-linear autonomous differential equations.
\item Convergence of dynamic motion primitive is guaranteed \cite{schaal2006dynamic}.  
\item Any arbitrary motion trajectory can be learned. 
\item Trajectories can be scaled in space as well as in time \cite{schaal2006dynamic,ijspeert2013dynamical}. 
\item These learned motion primitives can be initialized anywhere in the attractor space \cite{schaal2006dynamic,ijspeert2013dynamical}.

\end{itemize}
\end{frame}

\begin{frame}{Advantages of Dynamic Motion Primitves}
\begin{itemize}

\item On-line modifications in trajectory are possible \cite{schaal2006dynamic}. 
\item Obstacles can be avoided robustly \cite{park2008movement}. 
\item Re-planning is not needed unless an event causing major disturbance in the environment occurs \cite{park2008movement}.
\end{itemize}
\end{frame}

\begin{frame}{Related Work}
\begin{itemize}
\item Schaal et. al\cite{schaal2006dynamic} and Ijspeert et. al\cite{ijspeert2003learning,ijspeert2013dynamical} provide concrete theory and experimental results proving advantages of using DMPs.
\item Park et. al\cite{park2008movement} proposed method for obstacle avoiding using DMPs and potential fields. 
\item Lioutikov et. al , Nemec et. al , Park et.al have demonstrated motion sequencing by various methods. \cite{lioutikov2016learning,nemec2012action,park2008movement}
\item Software package implementing DMP framework is available in Robot Operating System.  
\end{itemize}
\end{frame}

\begin{frame}{Deficits}
\begin{itemize}
\item All the work done so far provide concrete base to implement DMP for robots. 
\item But it doesn't solve the problem of obtaining correct combination of simple motion primitives in order to do a particular task.  
\end{itemize}
\end{frame}

\bibliography{assignment_2}{}
\bibliographystyle{plain}



\end{document} 